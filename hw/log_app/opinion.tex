% Created 2021-03-15 Mon 23:29
% Intended LaTeX compiler: pdflatex
\documentclass[11pt]{article}
\usepackage[utf8]{inputenc}
\usepackage[T1]{fontenc}
\usepackage{graphicx}
\usepackage{grffile}
\usepackage{longtable}
\usepackage{wrapfig}
\usepackage{rotating}
\usepackage[normalem]{ulem}
\usepackage{amsmath}
\usepackage{textcomp}
\usepackage{amssymb}
\usepackage{capt-of}
\usepackage{hyperref}
\usepackage[spanish]{babel}
\usepackage[margin=1.5cm]{geometry}
\usepackage{biblatex}
\usepackage{arev}
\addbibresource{refs.bib}
\author{Edgar Quiroz}
\date{\today}
\title{Aplicación web para el proceso Educativo sobre el Logaritmo\\\medskip
\large Opinión del diseño}
\hypersetup{
 pdfauthor={Edgar Quiroz},
 pdftitle={Aplicación web para el proceso Educativo sobre el Logaritmo},
 pdfkeywords={},
 pdfsubject={},
 pdfcreator={Emacs 27.1 (Org mode 9.5)}, 
 pdflang={Spanish}}
\begin{document}

\maketitle
Revisa el artículo
\cite{Salas-Rueda_Gamboa-Rodríguez_Salas-Rueda_Salas-Rueda_2020}. Contesta la
siguiente pregunta utilizando los conceptos de usabilidad discutidos en la clase

\begin{itemize}
\item ¿Cuál es tu opinión sobre la Aplicación web para el proceso Educativo sobre
el Logaritmo (AEL)?

A grandes rasgo, me parece que es una aplicación que tomó muy bien en cuenta
su objetivo y sus usuarios, al ser extremadamente simple. Esto la hace ideal
para mostrar algunos aspectos aparentemente obvios del buen diseño. Por
ejemplo, una fuente de colores oscuros con un fondo blanco permiten buena
legibilidad. Aún así, hay varios aspecto que se podrían mejorar usando los
criterios vistos en clase.

Primero, hay que considerar la conetividad y la navegación. En las veces que
intenté usarla, actualizar la página tardó varios segundos, lo que hace
incómodo su uso. Más aún, tener que recargar la página entera para tener
interactividad en una aplicación tan simple se me hace extraño. Me parece que
ambos problemas serían solucionados, al menos en parte, remplazando los
componentes interactivos con alternativas que no requieran recargar la página.

Luego, respecto a la consistencia visual, la aplicación tiene tres fuentes
distintas y tres colores diferentes. Si bien las fuentes tienen diferentes
roles, estas podrían ser unificadas en una sola familia de fuentes. En cuanto
al color, no encuentro ningún patrón relavante, así que cambiar todo a un solo
color ayudaría con la consistencia. En otro aspecto de la consistencia, los
elementos se mueven levemente en cada recarga de la página. Esto se podría
solucionar con el punto anterior.

Finalmente, no me parece que la cara tenga algún propósito más que estético,
por lo que consideraría prudente retirarla.

A pesar de sus oportunidades para mejorar, el hecho de que sea tan simple hace
que estos detalles de diseño no afecten tanto la usabilidad, como se mostró en
las pruebas de usuario exhibidas en el artículo.
\end{itemize}

\printbibliography
\end{document}
